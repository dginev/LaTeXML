\documentclass{article}
\usepackage{xcolor}
\usepackage{adjustbox}
\usepackage{subcaption}
\usepackage{tcolorbox}
\newbox\tmp
\long\def\vboxsize#1#2{%
\setbox\tmp=\vbox{#2}%
\message{BOX size (#1): Width=\the\wd\tmp, Height=\the\ht\tmp, Depth=\the\dp\tmp}
#2}

\begin{document}

\vboxsize{min-outer}{
  \vboxsize{min-inner-1}{\vbox{word 1}}
  \vboxsize{min-inner-2}{\vbox{word 2}} % then par
  \par
  \vboxsize{min-inner-3}{\vbox{word 3}}}

\vboxsize{par}{\vboxsize{word}{Word}
  \vboxsize{sentence}{A much longer sentence is composed of many words.}
  \vboxsize{vmath}{$\sqrt{x}^2$}
  \vboxsize{vitems}{
    \begin{itemize}
      \item First item
      \item Second item
      \item Third item
    \end{itemize}}}

\begin{figure}
\vboxsize{tabular}{\begin{tabular}{l}
  \adjustbox{valign=t}{\vboxsize{subfig}{\begin{subfigure}{2cm}B\\C\\D\end{subfigure}}}
\end{tabular}}
\end{figure}

\def\mockenumerate{%
\begin{enumerate}
  \item Select a point from the restricted domain of the original function.  
  \item Reflect it across $y=x$ to form the \textit{Inverse Function Point}.  
  \item Solve algebraically for the inverse (which will usually produce $\pm$ roots).  
  \item Substitute the $x$-value of the Inverse Function Point; the correct root is the one whose $y$-value matches.  
\end{enumerate}}
\def\mockpars{%
\textcolor{brown}{\textbf{User:}}

I have 3 pencils, 2 pens, and 4 erasers.  How many things do I have?

\textcolor{brown}{\textbf{GPT-3:}}

You have 9 things. \textcolor{brown}{\textbf{[correct in 3 out of 3 trials]}}

\textcolor{brown}{\textbf{User:}}

I have 3 chickens, 2 ducks, and 4 geese.  How many things do I have? 

\textcolor{brown}{\textbf{GPT-3:}}

You have 10 animals total. \textcolor{brown}{\textbf{[incorrect in 3 out of 3 trials]}}}

minipage enumerate:
\vboxsize{minipage enumerate}{%
\begin{minipage}{\textwidth}
\mockenumerate
\end{minipage}}

minipage pars:
\vboxsize{minipage pars}{%
\begin{minipage}{\textwidth}
\mockpars
\end{minipage}}

\vboxsize{tcolorbox enumerate}{%
\begin{tcolorbox}[colback=blue!5,colframe=blue!40!black,title=Rule: Inverse Function Point Method]
  \mockenumerate
\end{tcolorbox}}

\vboxsize{tcolorbox pars}{%
\begin{tcolorbox}[colback=black!5!white,colframe=black!75!black,title=GPT-3 on Counting Items]
  \mockpars
\end{tcolorbox}}
\end{document}